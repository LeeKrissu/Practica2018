\documentclass{report}
\usepackage{ucs}
\usepackage[utf8x]{inputenc}
\usepackage[english,romanian]{babel}
\title{{\sc Raport asupra practicii: 25.06-06.07.2018}}
\author{Sarageaua Ana}
\date{\,}
\begin{document}
\maketitle

\tableofcontents

\chapter{Introducere}

 \item Implementarea algoritmului de sortare rapidă (quick sort). \newline

Quicksort este un celebru algoritm de sortare, dezvoltat de C. A. R. Hoare și care, în medie, efectuează O(n logn) comparații. De obicei, în practică, quicksort este mai rapid decât ceilalți algoritmi de sortare de complexitate O(n logn) deoarece bucla sa interioară are implementări eficiente pe majoritatea arhitecturilor și, în plus, în majoritatea implementărilor practice se pot lua, la proiectare, decizii ce ajută la evitarea cazului când complexitatea algoritmului este de {\displaystyle O(n^{2})} .

Quicksort efectuează sortarea bazându-se pe o strategie divide et impera. Astfel, el împarte lista de sortat în două subliste mai ușor de sortat. Pașii algoritmului sunt:

Se alege un element al listei, denumit pivot.

Se reordonează lista astfel încât toate elementele mai mici decât pivotul să fie plasate înaintea pivotului și toate elementele mai mari să fie după pivot. După această partiționare, pivotul se află în poziția sa finală.

Se sortează recursiv sublista de elemente mai mici decât pivotul și sublista de elemente mai mari decât pivotul.

O listă de dimensiune 0 sau 1 este considerată sortată.

Folosind metoda divide et impera problema iniţială va fi descompusă în subprobleme, astfel:

Pas1. Se rearanjează vectorul, determinându-se poziţia în care va fi mutat pivotul (m).

Pas2. Problema iniţială (sortarea vectorului iniţial) se descompune în două subprobleme prin descompunerea vectorului în doi subvectori: vectorului din stânga pivotului şi vectorul din dreapta
pivotului, care vor fi sortaţi prin aceeaşi metodă. Aceşti subvectori, la rândul lor, vor fi şi ei rearanjaţi şi împărţiţi de pivot în doi subvectori etc. 

Pas3. Procesul de descompunere în subprobleme ca continua până când, prin descompunerea vectorului în subvectori, se vor obţine vectori care conţin un singur element.

Subprogramele specifice algoritmului divide et impera vor avea următoarea semnificaţie:

 În subprogramul divizeaza() se va rearanja vectorul şi se va determina poziţia pivotului xm,
care va fi folosită pentru divizarea vectorului în doi subvectori :[xs, xm-1] şi [xm+1, xd].

 Subprogramul combina() nu mai este necesar, deoarece combinarea soluţiilor se face prin
rearanjarea elementelor în vector.

În subprogramul divizeaza() vectorul se parcurge de la ambele capete către poziţia în care trebuie
mutat pivotul.

Se vor folosi doi indici: i – pentru parcurgerea vectorului de la începutul lui către
poziţia pivotului (i se va incrementa) şi j – pentru parcurgerea vectorului de la sfârşitul lui către
poziţia pivotului (j se va decrementa). Cei doi indici vor fi iniţializaţi cu capetele vectorului (i=s,
respectiv j=d) şi se vor deplasa până când se întâlnesc, adică atât timp cât i<j. 

În momentul în
care cei doi indici s-au întâlnit înseamnă că operaţiile de rearanjare a vectorului s-au terminat şi
pivotul a fost adus în poziţia corespunzătoare lui în vectorul sortat. Această poziţie este i (sau j) şi
va fi poziţia m de divizare a vectorului.


Voi prezenta în continuare exemple unde se vor utiliza două versiuni pentru subprogramul
divizează().

Versiunea 1. 

3 4 1 5 2

Se folosesc variabilele logice: pi, pentru parcurgerea cu indicele i, şi pj, pentru
parcurgerea cu indicele j. Ele au valoarea: 1 – se parcurge vectorul cu acel indice, şi 0 – nu se 
10
parcurge vectorul cu acel indice; cele două valori sunt complementare.

Cei doi indici i şi j sunt iniţializaţi cu extremităţile vectorului(i=1, j=5) şi parcurgerea începe cu
indicele j (pi=0, pj=1)

Se compară elementul din poziţia i(4) cu elementul din poziţia j(3). Deoarece 4 este mai mare
decât 3, cele două valori se interschimbă, şi se schimbă şi modul de parcurgere (pi=0;pj=1 –
avansează inicele j).

Se compară elementul din poziţia i(3) cu elementul din poziţia j(5). Deoarece 3 este mai mic decât 5,
cele două valori nu se interschimbă, şi se schimbă modul de parcurgere (pi=0;pj=1 – avansează inicele
j).

Se compară elementul din poziţia i(3) cu elementul din poziţia j(1). Deoarece 3 este mai mare
decât 1, cele două valori se interschimbă, şi se schimbă modul de parcurgere (pi=0;pj=1 –
avansează inicele i).Cei doi indici fiind egali, algoritmul se termină.

Versiunea 2.

3 4 1 5 2 

Iniţial, cei doi indici i şi j sunt iniţializaţi cu extremităţile vectorului(i=1, j=5) şi pivotul are valoarea 3.

Elementul din poziţia i(3) nu este mai mic decât pivotul; indicele i nu avansează (i=1). Elementul
din poziţia j(2) nu este mai mare decât pivotul; indicele j nu avansează(j=5). Valorile din poziţiile i
şi j se interschimbă.

Elementul din poziţia i(2) este mai mic decât pivotul; indicele i avansează până la primul element
mai mare decât pivotul (i=2). Elementul din poziţia j(3) nu este mai mare decât pivotul; indicele j
nu avansează(j=5). Valorile din poziţiile i şi j se interschimbă.

Elementul din poziţia i(3) nu este mai mic decât pivotul; indicele i nu avansează (i=2). Elementul
din poziţia j(4) este mai mare decât pivotul; indicele j avansează până la primul element mai mic
decât pivotul (j=3). Valorile din poziţiile i şi j se interschimbă.

Elementul din poziţia i(1) este mai mic decât pivotul; indicele i avansează până la primul element
mai mare decât pivotul (i=4). Elementul din poziţia j(3) nu este mai mare decât pivotul; indicele j
nu avansează(j=3), algoritmul se termină.

\vskip 0.5cm
Raportul asupra practicii efectuate zilnic intre datele 25.06-06.07.2018. 

\chapter{Activități planificate}
\begin{enumerate}
\item  Luni, 25.06.2018 \newline
Aducerea la cunoștință a obiectivelor și cerințelor practicii de producție
\item  Marți, 26.06.2018 \newline
Configurarea sistemelor software pe calculatoare. 
\item  Miercuri, 27.06.2018 \newline
Studierea modului de lucru cu Git. Interfețe grafice de lucru cu Git (SmartGit).
\item  Joi, 28.06.2018 \newline
Studierea și practicarea LaTeX
\item  Vineri, 29.06.2018  \newline
Inițierea unei lucrări (descrierea unui algoritm, a unei teme agreate cu prof. coordonator)
\item  Luni, 02.07.2018  \newline
Lucrul asupra lucrării
\item  Marți, 03.07.2018  \newline
Lucrul asupra lucrării
\item  Miercuri, 04.07.2018  \newline
Prezentarea lucrărllor
\item  Joi, 05.07.2018  \newline
Prezentarea lucrărilor
\item  Vineri, 06.07.2018  \newline
Notarea finală a activității
\end{enumerate}
\chapter{25.06.2018}
Am desfăţurat următoarele activităţi:
\begin{itemize}
\item
Am identificat sursele pentru MikTeX, Git, SmartGit și BitBucket.
\begin{itemize}
\item
Am identificat sursele pentru MikTeX, Git, SmartGit și BitBucket.
\item
Am instalat, configurat pe calculatorul de lucru aplicațiile necesare:
\begin{itemize}
\item
MikTeX
\item
SmartGit
\item
Bitbucket
\end{itemize}
\item
Am instalat, configurat pe calculatorul de lucru aplicațiile necesare:
\begin{itemize}
\item
MikTeX
\item
SmartGit
\item
Bitbucket
\end{itemize}

\end{itemize}
\end{itemize}

\chapter{26.06.2018}
Studierea obiectivelor și cerințelor față de practica de producție. Clarificarea situațiilor incerte.
\chapter{27.06.2018}
Am studiat modul de lucru cu Git și interfața grafică de lucru cu Git (SmartGit).

\item Git este un sistem revision control care rulează pe majoritatea platformelor, inclusiv Linux, POSIX, Windows și OS X. Ca și Mercurial, Git este un sistem distribuit și nu întreține o bază de date comună. Este folosit în echipe de dezvoltare mari, în care membrii echipei acționează oarecum independent și sunt răspândiți pe o arie geografică mare.

Git este dezvoltat și întreținut de Junio Hamano, fiind publicat sub licență GPL și este considerat software liber.

Dezvoltarea Git a început după ce mai mulți developeri ai nucleului Linux au ales să renunțe la sistemul de revision control proprietar BitKeeper. Posibilitatea de a utiliza BitKeeper gratuit a fost retrasă după ce titularul drepturilor de autor a afirmat că Andrew Tridgell a încălcat licența BitKeeper prin acțiunile sale de inginerie inversă. La conferința Linux.Conf.Au 2005, Tridgell a demonstrat în timpul discursului său că procesul de inginerie inversă pe care l-a folosit a fost pur și simplu o sesiune telnet pe portul corespunzător al serverului BitKeeper și rularea comenzii help pe server.

Controversa a dus la o renunțarea rapidă la sistemul BitKeeper care a fost înlocuit cu un nou sistem intitulat Git construit special pentru scopul de revision control în cadrul proiectului Linux kernel. Dezvoltarea noului sistem a fost începută de Linus Torvalds în 3 aprilie 2005 pentru a fi anunțat câteva zile mai târziu (aprilie 6) pe lista de email a proiectului Linux kernel[28]. O zi mai târziu, noul sistem a început să fie folosit pentru dezvoltarea actuală de cod pentru proiectul Git. Primele operații merge a avut loc pe data de 18 aprilie. În data de 16 iunie, versiunea 2.6.12 Linux kernel a fost pusă în Git care continuă și în ziua de azi să fie sistemul revision control folosit de proiectul Linux kernel.

Tot în această perioadă, și tot cu scopul de a înlocui BitKeeper, a fost creat sistemul Mercurial.
\newline
\chapter{28.06.2018}
Am studiat și am practicat Latex.

\item LaTeX este un sistem de preparare a documentului, care permite tipărirea în format electronic cu ajutorul limbajului de programare TeX.

LaTeX a fost creat de Leslie Lamport în 1984 la SRI International și în timp a devenit principala metodă pentru programarea în TeX. Datorită capacităților de a programa în amănunt orice aspect care ține de publicarea unui material (articol, carte, tratat, broșură), LaTeX este folosit în general în mediu academic de către matematicieni, ingineri, etc, dar și în mediul comercial, datorită costurilor reduse de utilizare (LaTex și TeX sunt gratuite; TeX este eliberat de către creatorul său, Donald Knuth, în domeniu public). LaTeX permite programarea aspectelor necesare în desktop publishing, inclusiv tabele, figuri și imagini, referințe încrucișate, bibliografie și note bibliografice.

Din punct de vedere al limbajului de programare, LaTeX este un limbaj de programare de nivel-înalt, util în a accede la toate resursele limbajului TeX. Deoarece TeX este un limbaj de programare de nivel scăzut s-a dovedit a fi destul de dificil de utilizat de către utilizatorii comuni, motiv pentru care LaTeX a fost construit special pentru a permite oricărui utilizator să beneficieze de puterea limbajului TeX.

Versiunea curentă este (LaTeX2e). LaTeX, ca și TeX, este un program liber.
\newline
\chapter{29.06.2018}
Am inițiat o lucrare scrisă în Latex.
\chapter{30.06.2018}
Am continuat lucrul asupra temei alese.  

\item Implementarea algoritmului de sortare rapidă (quick sort). \newline
\chapter{02.07.2018}
Am continuat lucrul asupra temei și am terminat .
\chapter{03.07.2018}
Am continuat lucrul asupra temei și am terminat .
Prezentarea proiectului.
\chapter{04.07.2018}
Prezentarea proiectului.
\chapter{05.07.2018}
Prezentarea proiectului.
\chapter{06.07.2018}

Notarea finală a activității. Practica 2018.

\chapter{Concluzii}
Am invățat să lucrez cu Latex ,Git și BitBucket. Aici pot fi prezentate succint lucrurile învățate (câte ceva despre git, latex, bitbucket, etc - se pot adăuga secțiuni pentru fiecare subiect). 

Și acum să cităm unele dintre referințele noastre \cite{DUMMY:1} și \cite{book:25008}. La fel putem să cităm și alte cărți și surse online cum ar fi \cite{book:776133, book:1045183}. Alte exemple sunt în mostra / modelul unei lucrări de licență de pe site-ul facultății. 

% aici urmeaza declaratiile care fac posibila includerea bibliografiei in format bibtex. 

\bibliography{referinte} 
\bibliographystyle{ieeetr}

\end{document}
